\documentclass{article}[11pt]
\usepackage{longtable}
\usepackage{graphicx}
\usepackage{fix-cm}
\usepackage[margin=1in,paperwidth=8.5in,paperheight=11in]{geometry}
\usepackage[section]{placeins}
\usepackage{flafter}
\usepackage{amssymb}
\usepackage{amsmath}
\usepackage{etoolbox}
\usepackage{units}
\usepackage{multirow}
\usepackage{hyperref}
\usepackage{pdfcomment}
\usepackage{color}

\setcounter{topnumber}{3}
\setcounter{bottomnumber}{3}
\setcounter{totalnumber}{4}
\renewcommand{\topfraction}{0.9}
\renewcommand{\bottomfraction}{0.9}
\renewcommand{\textfraction}{0.1}
\renewcommand{\floatpagefraction}{0.7}

\newcommand{\rrangle}{\rangle\!\rangle} \newcommand{\llangle}{\langle\!\langle}
\newcommand{\ket}[1]{\ensuremath{\left|#1\right\rangle}}
\newcommand{\bra}[1]{\ensuremath{\left\langle#1\right|}}
\newcommand{\braket}[2]{\ensuremath{\left\langle#1|#2\right\rangle}}
\newcommand{\expec}[1]{\ensuremath{\left\langle#1\right\rangle}}
\newcommand{\ketbra}[2]{\ket{#1}\!\!\bra{#2}}
\newcommand{\braopket}[3]{\ensuremath{\bra{#1}#2\ket{#3}}}
\newcommand{\proj}[1]{\ketbra{#1}{#1}}
\newcommand{\sket}[1]{\ensuremath{\left|#1\right\rrangle}}
\newcommand{\sbra}[1]{\ensuremath{\left\llangle#1\right|}}
\newcommand{\sbraket}[2]{\ensuremath{\left\llangle#1|#2\right\rrangle}}
\newcommand{\sketbra}[2]{\sket{#1}\!\!\sbra{#2}}
\newcommand{\sbraopket}[3]{\ensuremath{\sbra{#1}#2\sket{#3}}}
\newcommand{\sproj}[1]{\sketbra{#1}{#1}}
\newcommand{\norm}[1]{\left\lVert#1\right\rVert}
\def\Id{1\!\mathrm{l}}
\newcommand{\Tr}[0]{\mathrm{Tr}}
\providecommand{\e}[1]{\ensuremath{\times 10^{#1}}}

%Command used for python automatic substitution
\newcommand{\putfield}[2]{#2}

\newtoggle{confidences}
\newtoggle{goodnessSection}
\toggletrue{confidences}
\toggletrue{goodnessSection}


\hypersetup{
  pdfinfo={ Author={pyGSTi},
Title={Brief GST report for $\mathcal{D}$},
Keywords={GST},
pyGSTi_Version={0.9.1-alpha},
opt_table_formats={[py, html, latex, ppt]},
opt_long_tables={False},
opt_table_class={dataTable},
opt_template_path={None},
opt_latex_cmd={pdflatex},
defaultDirectory={None},
L_germ_tuple_base_string_dict={Dict[[0, Gx]: {}, [0, Gy]: {}, [0, Gi]: {}, [0, GxGy]: {}, [0, GxGyGi]: {}, [0, GxGiGy]: {}, [0, GxGiGi]: {}, [0, GyGiGi]: {}, [0, GxGxGiGy]: {}, [0, GxGyGyGi]: {}, [0, GxGxGyGxGyGy]: {}, [1, Gx]: [Gx], [1, Gy]: [Gy], [1, Gi]: [Gi], [1, GxGy]: {}, [1, GxGyGi]: {}, [1, GxGiGy]: {}, [1, GxGiGi]: {}, [1, GyGiGi]: {}, [1, GxGxGiGy]: {}, [1, GxGyGyGi]: {}, [1, GxGxGyGxGyGy]: {}, [2, Gx]: [Gx]2, [2, Gy]: [Gy]2, [2, Gi]: [Gi]2, [2, GxGy]: [GxGy], [2, GxGyGi]: {}, [2, GxGiGy]: {}, [2, GxGiGi]: {}, [2, GyGiGi]: {}, [2, GxGxGiGy]: {}, [2, GxGyGyGi]: {}, [2, GxGxGyGxGyGy]: {}, [4, Gx]: [Gx]4, [4, Gy]: [Gy]4, [4, Gi]: [Gi]4, [4, GxGy]: [GxGy]2, [4, GxGyGi]: [GxGyGi], [4, GxGiGy]: [GxGiGy], [4, GxGiGi]: [GxGiGi], [4, GyGiGi]: [GyGiGi], [4, GxGxGiGy]: [GxGxGiGy], [4, GxGyGyGi]: [GxGyGyGi], [4, GxGxGyGxGyGy]: {}, [8, Gx]: [Gx]8, [8, Gy]: [Gy]8, [8, Gi]: [Gi]8, [8, GxGy]: [GxGy]4, [8, GxGyGi]: [GxGyGi]2, [8, GxGiGy]: [GxGiGy]2, [8, GxGiGi]: [GxGiGi]2, [8, GyGiGi]: [GyGiGi]2, [8, GxGxGiGy]: [GxGxGiGy]2, [8, GxGyGyGi]: [GxGyGyGi]2, [8, GxGxGyGxGyGy]: [GxGxGyGxGyGy]]},
max_length_list={[0, 1, 2, 4, 8]},
defaultBasename={None},
radius={0.0001},
minProbClip={1e-06},
fiducial_pairs={None},
mxBasis={gm},
minProbClipForWeighting={0.0001},
linlogPercentile={5},
probClipInterval={[-1000000.0, 1000000.0]},
memLimit={None},
weights={None},
objective={logl},
constrainToTP={True},
hessianProjection={std}  }
}


\begin{document}

\title{Brief GST report for $\mathcal{D}$}
\date{\vspace{-1cm}\today}
%\author{}

\begingroup
\let\center\flushleft
\let\endcenter\endflushleft
\maketitle
\endgroup

\section{Overview}
This report presents a brief gate-set tomography (GST) analysis of a dataset called ``$\mathcal{D}$''.   It is intended for folks who are already familiar with GST output tables and plots.  If this isn't you, you should have GST generate a ``full report'' which give much more detail about what GST is and what the results in this ``brief report'' mean.

\section{Goodness-of-model}
\iftoggle{goodnessSection}{
Table \ref{progressTable} shows how well GST was able to fit you data through its iterations.
%, and Figure \ref{bestEstimateColorBoxPlot} shows in more detail the goodness of fit quantity for the final (best) gateset estimate.

\begin{table}[h]
\begin{center}
\begin{tabular}[l]{|c|c|c|c|c|c|c|c|c|}
\hline
L & $2\Delta\log(\mathcal{L})$ & $k$ & $2\Delta\log(\mathcal{L})-k$ & $\sqrt{2k}$ & $p$ & $N_s$ & $N_p$ & Rating \\ \hline
0 & 53.74648 & 61 & -7.253515 & 11.04536 & 0.73 & 92 & 31 & $\bigstar\bigstar\bigstar\bigstar\bigstar$ \\ \hline
1 & 53.7465 & 61 & -7.253495 & 11.04536 & 0.73 & 92 & 31 & $\bigstar\bigstar\bigstar\bigstar\bigstar$ \\ \hline
2 & 106.745 & 137 & -30.25503 & 16.55295 & 0.97 & 168 & 31 & $\bigstar\bigstar\bigstar\bigstar\bigstar$ \\ \hline
4 & 354.3548 & 410 & -55.64517 & 28.63564 & 0.98 & 441 & 31 & $\bigstar\bigstar\bigstar\bigstar\bigstar$ \\ \hline
8 & 691.0838 & 786 & -94.91617 & 39.64846 & 0.99 & 817 & 31 & $\bigstar\bigstar\bigstar\bigstar\bigstar$ \\ \hline
\end{tabular}

\caption{\textbf{Comparison between the computed and expected $2\Delta\log{\mathcal{L}}$ for different values of $L$}.  $N_S$ and $N_p$ are the number of gate strings and parameters, respectively.  $2\Delta\log{\mathcal{L}}$ is expected to lie within $[k-\sqrt{2k},k+\sqrt{2k}]$ where $k = N_s-N_p$. $P$ is the p-value derived from a $\chi^2_k$ distribution (i.e.~when $P <$ some threshold like $0.05$ there is grounds for rejecting the GST fit). The rating from 1 to 5 stars gives a very crude indication of goodness of fit as explained in the text.\label{progressTable}}
\end{center}
\end{table}


%\begin{figure}
%\begin{center}
%\XXputfield{bestEstimateColorBoxPlot}{Color box plot for best gateset}
%\caption{\XXputfield{tt_bestEstimateColorBoxPlot}{}\textbf{\XXputfield{objective}{??} values for every individual experiment in the dataset}.  Each pixel represents a single experiment (gate sequence), and its color indicates whether GST was able to fit the corresponding frequency well.  Blue is typical; dark red squares indicating $\chi^2>10$ should appear only once per 638 experiments on average.  Square blocks of pixels correspond to base sequences (arranged vertically by germ and horizontally by length); each pixel within a block corresponds to a specific choice of pre- and post-fiducial sequences.  See text for further details.\label{bestEstimateColorBoxPlot}}
%\end{center}
%\end{figure}

}{ (You need to supply the GST Results object with more data in order to compute a goodness-of-fit section.) }

\FloatBarrier

\section{Gateset Estimates}
Below is a collection of GST's primary result tables, ordered in what we think is a natural way.  These tables all pertain to the best-estimate gateset, that is, the one corresponding to the final iteration of the GST algorithm.

\subsection{Derived Quantities}

\begin{table}[h]
\begin{center}
\begin{tabular}[l]{|c|c|c|c|c|}
\hline
Gate & \begin{tabular}{c}Process\\Infidelity\end{tabular} & \begin{tabular}{c}$\nicefrac{1}{2}$ Trace\\Distance\end{tabular} & $\nicefrac{1}{2}$ $\Diamond$-Norm & \begin{tabular}{c}Frobenius\\Distance\end{tabular} \\ \hline
Gi & $ \begin{array}{c} 0.039929 \\ \pm 0.00257 \end{array} $ & $ \begin{array}{c} 0.0401 \\ \pm 0.002571 \end{array} $ & $ \begin{array}{c} 0.040106 \\ \pm 0.00257 \end{array} $ & $ \begin{array}{c} 0.092986 \\ \pm 0.00596 \end{array} $ \\ \hline
Gx & $ \begin{array}{c} 0.037847 \\ \pm 0.002167 \end{array} $ & $ \begin{array}{c} 0.037919 \\ \pm 0.002167 \end{array} $ & $ \begin{array}{c} 0.03792 \\ \pm 0.002165 \end{array} $ & $ \begin{array}{c} 0.087746 \\ \pm 0.00499 \end{array} $ \\ \hline
Gy & $ \begin{array}{c} 0.037738 \\ \pm 0.002227 \end{array} $ & $ \begin{array}{c} 0.037804 \\ \pm 0.002228 \end{array} $ & $ \begin{array}{c} 0.037804 \\ \pm 0.002229 \end{array} $ & $ \begin{array}{c} 0.0875 \\ \pm 0.005127 \end{array} $ \\ \hline
\end{tabular}

\vspace{2em}
\begin{tabular}[l]{|c|c|}
\hline
Gate & Error Generator \\ \hline
Gi & $ \left(\!\!\begin{array}{cccc}
0 & 0 & 0 & 0 \\ 
-0.001 & -0.0569 & -0.0066 & -0.0093 \\ 
0.0004 & 0.0011 & -0.0543 & 0.0041 \\ 
0.0011 & 3\e{-5} & 0.0008 & -0.0529
 \end{array}\!\!\right) $
 \\ \hline
Gx & $ \left(\!\!\begin{array}{cccc}
0 & 0 & 0 & 0 \\ 
0.0003 & -0.0508 & -0.0051 & -0.0057 \\ 
0.0002 & 0.0003 & -0.0523 & -0.0023 \\ 
-0.0003 & -0.0003 & -0.0001 & -0.0523
 \end{array}\!\!\right) $
 \\ \hline
Gy & $ \left(\!\!\begin{array}{cccc}
0 & 0 & 0 & 0 \\ 
0.0001 & -0.0524 & 0.0065 & 0.0005 \\ 
0.0003 & 0.0011 & -0.05 & 0.0012 \\ 
0.0002 & 0.0001 & -0.0042 & -0.0525
 \end{array}\!\!\right) $
 \\ \hline
\end{tabular}

\caption{\textbf{Comparison of GST estimated gates to target gates}.  This table presents, for each of the gates, three different measures of distance or discrepancy from the GST estimate to the ideal target operation.  See text for more detail.  The second table lists the ``Error Generator'' for each gate, which is the Lindbladian $\mathbb{L}$ that describes \emph{how} the gate is failing to match the target.  This error generator is defined by the equation $\hat{G} = G_{\mathrm{target}}e^{\mathbb{L}}$. \label{bestGatesetVsTargetTable}}
\end{center}
\end{table}

\begin{table}[h]
\small
\begin{center}
\begin{tabular}[l]{|c|c|c|c|c|c|}
\hline
Gate & Eigenvalues & Fixed pt & Rotn. axis & Diag. decay & Off-diag. decay \\ \hline
Gi & $ \begin{array}{c}
0.9487 \\ 
0.9458e^{i0.0} \\ 
0.9458e^{-i0.0} \\ 
1
 \end{array} $
 $\pm$ $ \begin{array}{c}
0.019 \\ 
0.013e^{i0.7} \\ 
0.013e^{i0.7} \\ 
0
 \end{array} $
 & $ \begin{array}{c}
0.999 \\ 
-0.0223 \\ 
0.0093 \\ 
0.0206
 \end{array} $
 & $ \begin{array}{c}
0 \\ 
0.9348 \\ 
-0.1473 \\ 
-0.3233
 \end{array} $
 & $ \begin{array}{c} 0.05127 \\ \pm 0.019012 \end{array} $ & $ \begin{array}{c} 0.054222 \\ \pm 0.010132 \end{array} $ \\ \hline
Gx & $ \begin{array}{c}
0.9491e^{i1.6} \\ 
0.9491e^{-i1.6} \\ 
0.9505 \\ 
1
 \end{array} $
 $\pm$ $ \begin{array}{c}
0.0047e^{i0.7} \\ 
0.0047e^{i0.7} \\ 
0.0056 \\ 
0
 \end{array} $
 & $ \begin{array}{c}
1 \\ 
0.0064 \\ 
3\e{-5} \\ 
0.0002
 \end{array} $
 & $ \begin{array}{c}
0 \\ 
-1 \\ 
3\e{-5} \\ 
-0.0003
 \end{array} $
 & $ \begin{array}{c} 0.049514 \\ \pm 0.005638 \end{array} $ & $ \begin{array}{c} 0.050938 \\ \pm 0.002912 \end{array} $ \\ \hline
Gy & $ \begin{array}{c}
0.9489e^{i1.6} \\ 
0.9489e^{-i1.6} \\ 
0.9512 \\ 
1
 \end{array} $
 $\pm$ $ \begin{array}{c}
0.0046e^{i0.7} \\ 
0.0046e^{i0.7} \\ 
0.0057 \\ 
0
 \end{array} $
 & $ \begin{array}{c}
1 \\ 
1\e{-5} \\ 
0.0067 \\ 
-0.0002
 \end{array} $
 & $ \begin{array}{c}
0 \\ 
-0.0054 \\ 
1 \\ 
-0.0012
 \end{array} $
 & $ \begin{array}{c} 0.048788 \\ \pm 0.005692 \end{array} $ & $ \begin{array}{c} 0.051084 \\ \pm 0.00302 \end{array} $ \\ \hline
\end{tabular}

\vspace{2em}
\begin{tabular}[l]{|c|c|c|c|c|}
\hline
\multirow{2}{*}{Gate} & \multirow{2}{*}{Angle} & \multicolumn{3}{c|}{Angle between Rotation Axes} \\ \cline{3-5}
 & & Gi & Gx & Gy \\ \hline
Gi & $ \begin{array}{c}(0.000625 \\ \pm 0.002733)\pi \end{array} $ &  & $ \begin{array}{c}(0.884316 \\ \pm 0.485651)\pi \end{array} $ & $ \begin{array}{c}(0.548542 \\ \pm 0.631029)\pi \end{array} $ \\ \hline
Gx & $ \begin{array}{c}(0.500356 \\ \pm 0.001226)\pi \end{array} $ & $ \begin{array}{c}(0.884316 \\ \pm 0.485651)\pi \end{array} $ &  & $ \begin{array}{c}(0.498284 \\ \pm 0.001851)\pi \end{array} $ \\ \hline
Gy & $ \begin{array}{c}(0.50007 \\ \pm 0.001144)\pi \end{array} $ & $ \begin{array}{c}(0.548542 \\ \pm 0.631029)\pi \end{array} $ & $ \begin{array}{c}(0.498284 \\ \pm 0.001851)\pi \end{array} $ &  \\ \hline
\end{tabular}

\caption{\textbf{Eigen-decomposition of estimated gates}.  Each estimated gate is described in terms of: (1) the eigenvalues of the superoperator; (2) the gate's fixed point (as a vector in $\mathcal{B}(\mathcal{H})$, in the Pauli basis); (3)  the axis around which it rotates, as a vector in $\mathcal{B}(\mathcal{H})$; (4) the angle of the rotation that it applies; (5) the decay rate along the axis of rotation (``diagonal decay''); (6) the decay rate perpendicular to the axis of rotation (``off-diagonal decay''); and (7) the angle between each gate's rotation axis and the rotation axes of the other gates.  ``X'' indicates that the decomposition failed or couldn't be interpreted. \label{bestGatesetDecompTable}}
\end{center}
\end{table}

\FloatBarrier

\subsection{Raw Estimates}

\begin{table}[h]
\begin{center}
\begin{tabular}[l]{|c|c|c|c|}
\hline
Operator & Hilbert-Schmidt vector (Pauli basis) & 95\% C.I. $\nicefrac{1}{2}$-width & Matrix \\ \hline
$\rho_{0}$ & $ \begin{array}{c}
0.7071 \\ 
0.0016 \\ 
0.0003 \\ 
0.6396
 \end{array} $
 & $ \begin{array}{c}
0 \\ 
0.0042 \\ 
0.0044 \\ 
0.0045
 \end{array} $
 & $ \left(\!\!\begin{array}{cc}
0.9523 & 0.0012e^{-i0.2} \\ 
0.0012e^{i0.2} & 0.0477
 \end{array}\!\!\right) $
 \\ \hline
$E_{0}$ & $ \begin{array}{c}
0.7078 \\ 
-0.0002 \\ 
-0.0031 \\ 
-0.6407
 \end{array} $
 & $ \begin{array}{c}
0.0014 \\ 
0.0047 \\ 
0.0048 \\ 
0.0045
 \end{array} $
 & $ \left(\!\!\begin{array}{cc}
0.0475 & 0.0022e^{i1.6} \\ 
0.0022e^{-i1.6} & 0.9535
 \end{array}\!\!\right) $
 \\ \hline
\end{tabular}

\caption{\textbf{The GST estimate of the SPAM operations}.  Compare to Table \ref{targetSpamTable}.\label{bestGatesetSpamTable}}
\end{center}
\end{table}

\begin{table}[h]
\begin{center}
\begin{tabular}[l]{|c|c|c|}
\hline
 & $E_{0}$ & $E_C$ \\ \hline
$\rho_{0}$ & $ \begin{array}{c} 0.090701 \\ \pm 0.00618 \end{array} $ & $ \begin{array}{c} 0.9092 \\ \pm 0.006179 \end{array} $ \\ \hline
\end{tabular}

\caption{\textbf{GST estimate of SPAM probabilities}.  Computed by taking the dot products of vectors in Table \ref{bestGatesetSpamTable}.\label{bestGatesetSpamParametersTable}}
\end{center}
\end{table}

\begin{table}[h]
\begin{center}
\begin{tabular}[l]{|c|c|c|}
\hline
Gate & Superoperator (Pauli basis) & 95\% C.I. $\nicefrac{1}{2}$-width \\ \hline
Gi & $ \left(\!\!\begin{array}{cccc}
1 & 0 & 0 & 0 \\ 
-0.001 & 0.9447 & -0.0063 & -0.0088 \\ 
0.0004 & 0.0011 & 0.9471 & 0.0038 \\ 
0.0011 & 3\e{-5} & 0.0007 & 0.9485
 \end{array}\!\!\right) $
 & $ \left(\!\!\begin{array}{cccc}
0 & 0 & 0 & 0 \\ 
0.0022 & 0.0056 & 0.006 & 0.0061 \\ 
0.0022 & 0.0058 & 0.0056 & 0.006 \\ 
0.0021 & 0.0056 & 0.0059 & 0.0051
 \end{array}\!\!\right) $
 \\ \hline
Gx & $ \left(\!\!\begin{array}{cccc}
1 & 0 & 0 & 0 \\ 
0.0003 & 0.9505 & -0.0049 & -0.0054 \\ 
0.0003 & 0.0002 & 0.0001 & -0.9491 \\ 
0.0002 & 0.0003 & 0.949 & -0.0022
 \end{array}\!\!\right) $
 & $ \left(\!\!\begin{array}{cccc}
0 & 0 & 0 & 0 \\ 
0.0022 & 0.0056 & 0.006 & 0.0057 \\ 
0.0021 & 0.0046 & 0.0059 & 0.0029 \\ 
0.002 & 0.0045 & 0.0029 & 0.0038
 \end{array}\!\!\right) $
 \\ \hline
Gy & $ \left(\!\!\begin{array}{cccc}
1 & 0 & 0 & 0 \\ 
0.0002 & 0.0001 & -0.004 & 0.9488 \\ 
0.0003 & 0.0011 & 0.9512 & 0.0011 \\ 
-0.0001 & -0.949 & -0.0062 & -0.0005
 \end{array}\!\!\right) $
 & $ \left(\!\!\begin{array}{cccc}
0 & 0 & 0 & 0 \\ 
0.002 & 0.006 & 0.006 & 0.003 \\ 
0.0022 & 0.0048 & 0.0057 & 0.0047 \\ 
0.002 & 0.003 & 0.0053 & 0.0035
 \end{array}\!\!\right) $
 \\ \hline
\end{tabular}

\caption{\textbf{The GST estimate of the logic gate operations}.  Compare to Table \ref{targetGatesTable}.\label{bestGatesetGatesTable}}
\end{center}
\end{table}

\end{document}